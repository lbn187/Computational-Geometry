\documentclass[12pt]{article}
\usepackage{CJK}
\usepackage{bm}
\usepackage{framed}
\begin{document}

\begin{CJK*}{GBK}{song}
    \title{计算几何笔记}
    \author{李泊宁}
    \maketitle
    \section{Convex Hull}\
        \subsection{Extreme Points(EP)}\
            \par Let point set $S=\{p_1,p_2,\cdots ,p_n\}\subset \varepsilon^2$
            \par A point $p$ of $S$ is called an \textbf{extreme points(EP)} iff there exists a line $L$ through $p$ such that all point of $S$ lie on the \textbf{same side} of $L$.
            \par A polygon is \textbf{convex} iff every vertex is extreme.
        \subsection{To-Left Test}\
            \par $ToLeft(a,b,s)$ is true if s lies left to $\overrightarrow{ab}$, otherwise it's faluse.
            \paragraph{Implementation}$ToLeft(p,q,s)\to Area2(p,q,s)>0\to \left |\begin{array}{ccc}
                                                                            px & py & 1 \\
                                                                            qx & qy & 1 \\
                                                                            sx & sy & 1 
                                                                          \end{array}\right | >0
            $。避免除法和三角函数,从而减小误差。
        \subsection{In-Triangle}\
            \par $InTriangle(p,q,r,s)$ is true iff $ToLeft(p,q,s),ToLeft(q,r,s),ToLeft(r,p,s)$ are all true.
            \par We can use In-Triangle test to excluding non-extreme points in $O(n^4)$ time.
        \subsection{Extreme Edges(EE)}\
            \par Extreme Edges are edge $e$ that all points in $S$ lie to the \textbf{same side} $e$.
            \par 枚举所有有向线段$overrightarrow{ab}$,检查所有点,达到$O(n^3)$复杂度。
        \subsection{In-Convex-Polygon Test(ICPT)}\
            \paragraph{静态}给定$n$个点的凸包,$m$次询问,每次用二分缩小范围,直到缩小到一个三角形内,用To-Left Test即可,时间复杂度$O(m\log n)$。
            \par 如果凸包不是静态的,直接用$n$遍To-Left Test判断即可。
        \subsection{Decrease and Conquer}\
            \par 依次加点并进行ICPT,如果ICPT为真,那么凸包不会改变。
            \par Otherwise, for a point $x$ exterior to the hull, there are two tangent points $s$ and $t$. To update the hull, we simply throw $ts$ away and connect $x$ with $s$ and $t$ resp.
            \par Given $x$, all current EP's can be classified into \textbf{4 types}. For each $v$, consider $ToLeft(x,v,pre(v))$ and $ToLeft(x,v,suc(v))$。
            \par 那么,$L+L$——$s$;$R+R$——$t$;$R+L$:$R+L$——$st$;$L+R$——$ts$。
            \par 动态维护凸包即可,时间复杂度$O(n^2)$。
        \subsection{Jarvis March}\
            \par The stategy is finding the next EP before attaching it to a \textbf{partial convex hull}. Time complixity is $O(n^2)$.
            \par All directed EE's form a CCW cycle. Each pair of consecutive EE's *share* an endpoint.
            \par Let $ik$ be an EE, then get next EE $ks$ which has \textbf{minimum} left-turn.
            \par We can use To-Left Test to make this step $O(n)$.
            \par What's more, we can let the \textbf{lowest-then-leftmost(LTL)} point as the start point.
        \subsection{Lower Bound}\
            \par Convex Problem's lower bound is $O(n\log n)$.
            \par Consider a sequence $\{a_1,a_2,\cdots, a_n\}$, we make $n$ points in the dimention and let $a_i$ trans to $(a_i,a_i^2)$ and try to find the conver of the point set. We can find convex problem is at least a sorting problem, which lower bound is $O(n\log n)$. So we can learn that convex problem can't be easier than sorting problem.
        \subsection{Graham Scan}\
            \subsubsection{Presorting}\
                \par 找到LTL点$p_1$,然后对其他所有点按照对$p_1$的极角序排序并标号(排序不直接求角,会有很大精度误差,直接将cmp函数改为ToLeft测试即可)
                \par 将排好序后的$p_1,p_2$加到栈$S$中,将其余点$\{p_3,p_4,\cdots,p_n\}$依次判进行如下的判断。
            \subsubsection{Working}\
                \par 假设判断的点是$p_i$在栈的第二个点和第一个点的向量的左端,即$ToLeft(S_1,S_0,p_i)$为真,将$p_i$加到$S$ 栈中,否则将$S$ 的栈顶弹出,急需判断$p_i$。因为$S$中元素肯定不会少于$2$个,故不用判断$S$中元素是否大于等于$2$个。
            \subsubsection{Correctness}\
                \par 使用数学归纳法。假设按极角排好序后,到$k$个点时已经求出了前$k$个点的凸包,现在证明加入第$k+1$个点后求出的凸包是正确的,根据这个点在左手边或右手边分情况讨论分析,可以画图证明。
                \par 为什么要排序?我们可以画不断往左转的图案,容易发现不排序就会挂掉。
            \subsubsection{Trick}\
                \par 一般来说,Graham扫描法的时间复杂度是$O(n\log n)$,但如果点集有序,可以达到线性复杂度。
                \par 使用Graham扫描法判断过的所有边构成一个平面图,而$n$个点的平面图边数不超过$3n$级别,故排好序后算法是线性的。
                \par 如果一开始已经按X轴排好序,我们不需要做极角排序就可以在$O(n)$的时间内求出凸包。考虑在Y轴的正负无穷处各有一个极点,从极点处看这些在X轴排好序的点的极角序也是排好的。那么可以用Graham扫描法做一遍\textbf{上凸包},再做一遍\textbf{下凸包},把两个凸包\textbf{合并}即可线性时间得到凸包。
        \subsection{Divide-and-Conquer Merge}\
            \par 可以使用分治的方法,如果能线性合并两个凸包,那么总时间复杂度为$O(n\log n)$。
            \par 容易发现,如果选定一个$Kernal$,那么两个凸包上所有点形成一个star-shaped polygon。如何找Kernel?任取凸包上的三个点,求重心即可。
            \par 简单的情况是选定的Kernal在两个凸包中(用ICPT判断),那么两个凸包中所有的点是环状有序的,将它们环状归并,归并后再调用Graham扫描法就可以再线性时间内得到合并后的凸包。
            \par 复杂的情况是选定的Kernal不在第二个凸包$P_2$中,那么求出Kernal关于$P_2$的切点$s,t$,那么容易发现$ts$中的点不在新的凸包中,只要考虑$st$中的点即可。这种做法也是线性的,符合要求。
        \subsection{Divide-and-Conquer Prepocessing}\
            \par 先将凸包按X轴排序,那么每次只需要合并左右两个凸包即可。
            \par 容易发现,合并后新的凸包大多数边都在原来的两个凸包中,只有两条边除外。也就是说,我们只要找到这两条边和对应的四个切点即可。但找切点没有这么简单,切点并不是最高点和最低点。
            \par 我们只考虑求上切线,求下切线的方法类似。容易发现,左凸包的上切点一定满足两次ToLeft测试是LL,而右凸包的上切点一定满足两次ToLeft测试是RR。
            \par 我们先将左凸包的rightmost vertex和右凸包的leftmost vertex连接,然后使用Zig-Zag方法,如果不满足左边的LL或者右边的RR,就朝对应方向移动一步,直到满足条件为止。这样总的走步数是线性的,总时间复杂度$O(n\log n)$。
        \subsection{Pegeneracy}\
            \par 有一些特殊情况是比较难以处理的,需要使用高妙的方法。
            \par 比如三点共线、两点某一维坐标相同、四点共圆等。
            \par 三维凸包的复杂度和二维凸包相同,而四维凸包需要$O(n^2)$求解。
\end{CJK*}
\end{document}
